
\documentclass{include/thesisclass3}

\SelectLanguage{ngerman}
\usepackage{float}


% Titlepage settings
\newcommand{\praktikum}{Praktikum moderne Physik}
\newcommand{\autora}{Jens Schäfer}
\newcommand{\autorb}{Jan van der Linden}
\newcommand{\maila}{schaferjens92@gmail.com}
\newcommand{\mailb}{jan.vdlinden95@gmail.com}
\newcommand{\topic}{Versuch}
\newcommand{\ptime}{Datum}


% Shortcuts
\newcommand{\cc}{\cdot}
\newcommand{\rk}{\rangle}
\newcommand{\lk}{\langle}
\newcommand{\df}{\rightarrow}
\newcommand{\la}{\lambda}
\newcommand{\dd}{{\rm d}}
\newcommand{\ehm}{\mathbbm{1}}
\newcommand{\p}{\partial}
\newcommand{\soll}{\overset{!}{=}}
\newcommand{\D}{\Delta}
\newcommand{\eps}{\epsilon}
\newcommand{\vektor}[3]{\begin{pmatrix} #1 \\ #2 \\ #3 \end{pmatrix}}
\newcommand{\vektorz}[2]{\begin{pmatrix} #1 \\ #2 \end{pmatrix}}
\newcommand{\Mat}[9]{\begin{pmatrix}#1&#2&#3\\#4&#5&#6\\#7&#8&#9\end{pmatrix}}
\newcommand{\Matz}[4]{\begin{pmatrix}#1&#2\\#3&#4\end{pmatrix}}



\begin{document}

	\FrontMatter
	% coordinates for background border
\newcommand{\diameter}{20}
\newcommand{\xone}{-15}
\newcommand{\xtwo}{160}
\newcommand{\yone}{15}
\newcommand{\ytwo}{-253}




\begin{titlepage}
    % background border
    \begin{tikzpicture}[overlay]
    \draw[color=gray]
            (\xone mm, \yone mm)
      -- (\xtwo mm, \yone mm)
    arc (90:0:\diameter pt)
      -- (\xtwo mm + \diameter pt , \ytwo mm)
        -- (\xone mm + \diameter pt , \ytwo mm)
    arc (270:180:\diameter pt)
        -- (\xone mm, \yone mm);
    \end{tikzpicture}



    % KIT image and sign for faculty of physics
    \begin{textblock}{10}[0,0](4.5,2.5)
        \includegraphics[width=.25\textwidth]{include/kitlogo.pdf}
    \end{textblock}
    

    % horizontal line
    \begin{textblock}{10}[0,0](4.2,3.1)
        \begin{tikzpicture}[overlay]
        \draw[color=gray]
                (\xone mm + 5 mm, -12 mm)
          -- (\xtwo mm + \diameter pt - 5 mm, -12 mm);
        \end{tikzpicture}
    \end{textblock}



    % begin of text part
    \changefont{phv}{m}{n}    % helvetica
    \centering



    % thesis topic (en and ge)
    \vspace*{3cm}
    \Huge\praktikum\\



    % author name and institute
    \vspace*{5cm}
    
    \huge\topic\\






    % examiners (Referenten)
    \vspace*{3cm}
    \Large
    \begin{center}
        \begin{tabular}[ht]{l c l } 
  \autora & \hfill & \textit{\maila} \\
\autorb & \hfill & \textit{\mailb} \\
        
        \end{tabular}
    \end{center}



    % working time
    \vspace{2cm}
    \begin{center}
        \large{Durchgeführt am}: \ptime
    \end{center}



    % lowest text blocks concerning the KIT
    \begin{textblock}{10}[0,0](4,16.8)
        \tiny{KIT -- Universität des Landes Baden-Württemberg und nationales %
              Forschungszentrum in der Helmholtz-Gemeinschaft}
    \end{textblock}
    \begin{textblock}{10}[0,0](14,16.75)
        \large{\textbf{www.kit.edu}}
    \end{textblock}
\end{titlepage}

	\tableofcontents                  
	\newpage
	\MainMatter

%Protokollstart

\chapter{Theoretische Grundlagen}
\section{Ziel des Experiments}
In diesem Experiment soll anhand eines Modells untersucht werden, wie ein Antibiotikum auf eine Zellmembran wirkt. Dazu wird ein Präparat einer Lipid Doppelmembran angefertigt und einem Antibiotikum (Gramicidin A) ausgesetzt. Durch das Antibiotikum werden Ionenkanäle in der Membran gebildet und somit Kationenfluss ermöglicht. Für eine Zelle bedeutet dies, dass ihr Potential zusammenbricht und sie stirbt.\\
\\
Hier soll die Generierung einer solchen Membran beobachtet werden und das Entstehen sowie die Beschaffenheiten der Ionenkanäle näher untersucht werden.


\section{Black Lipid Membran Methode}

\section{Gramicidin A}
%Kap 5

\section{Aufbau des Experiments}



\section{Generierung der Membran}
Als erster Versuchsschritt soll eine Membran generiert werden. Dazu wird eine geringe Menge an Glycerol Monooleat in Tetradecan auf die Öffnung gebracht. Dadurch wird eine Lipide Doppelmembran generiert. Dies soll optisch beobachtet werden; die Farbe der Membran verändert sich, je nach Dicke bis sie komplett schwarz wird.\\
Für die fertige Membran kann die Leitfähigkeit sowie die Kapazität und die Dicke bestimmt werden. \\Die Leitfähigkeit $G$ kann über den gemessenen Strom durch die Membran und die angelegte Spannung bestimmt werden über $G = \frac{I}{U}$. \\Um die Kapazität zu bestimmen kann angenommen werden, dass die Doppelmembran  sich im Stromkreislauf wie ein Plattenkondenstator verhält. Daher ist der übliche Zusammenhang für einen Plattenkondensator gültig.
\begin{equation}
C = e_0e_m \frac{A}{d}
\end{equation}

% Dicke, spez. Membrankapazität, Membranwiderstand, Durchbruchspannung, max. E-Feld


\section{Einzelkanal Messungen}
Gibt man nur sehr geringe Mengen Gramicidin A in die Lösung so bilden sich auch nur sehr wenige Kanäle in der Membran. Dies wird durch einen quantisierten Stromverlauf sichtbar womit die Anzahl der aktiven Kanäle und auch der Stromfluss pro Kanal bestimmbar sind.\\
Ebenso kann durch die Aufnahme eines Stromhistogramms die Lebensdauer eines Kanals bestimmt werden.

%Ohms Gesetz

\section{Multikanal Messungen}
Erhöht man die Konzentration des Gramicidin A so öffnen und schließen sich viel mehr Kanäle und eine Vermessung der einzelnen Kanäle ist nicht mehr möglich. Jedoch besteht die Möglichkeit Aussagen über das Öffnen und Schließen der Kanäle, also die Lebenszeit durch eine Autokorrelationsfunktion zu treffen.\\
Dazu soll ein Stromhistogramm aufgenommen werden und über Autokorrelation verglichen werden. Dazu verschiebt man den Verlauf des Histogramms in der Zeit und berechnet die Korrelation des verschobenen Verlaufs mit dem ursprünglichen Verlauf. Ein hoher Korrelationswert deutet auf eine, sich wiederholende Struktur hin, woraus die Halbwertszeit bestimmt werden kann.

\section{Weitere Analysen}

% Bestimmung des Dissoziation der Ratenkoeffizienten des Antibiotikum (?????)
% Kanalleitfähigkeit
% Teilchenfluss



\chapter{Versuchsauswertung}


\end{document}
