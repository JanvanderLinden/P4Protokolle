
\documentclass{include/thesisclass3}

\SelectLanguage{ngerman}
\usepackage{float}


% Titlepage settings
\newcommand{\praktikum}{Praktikum moderne Physik}
\newcommand{\autora}{Jens Schäfer}
\newcommand{\autorb}{Jan van der Linden}
\newcommand{\maila}{ugecd@student.kit.edu}
\newcommand{\mailb}{jan.vdlinden95@gmail.com}
\newcommand{\topic}{Quantenradierer}
\newcommand{\ptime}{22. Mai 2017}


% Shortcuts
\newcommand{\cc}{\cdot}
\newcommand{\rk}{\rangle}
\newcommand{\lk}{\langle}
\newcommand{\df}{\rightarrow}
\newcommand{\la}{\lambda}
\newcommand{\dd}{{\rm d}}
\newcommand{\ehm}{\mathbbm{1}}
\newcommand{\p}{\partial}
\newcommand{\soll}{\overset{!}{=}}
\newcommand{\D}{\Delta}
\newcommand{\eps}{\epsilon}
\newcommand{\vektor}[3]{\begin{pmatrix} #1 \\ #2 \\ #3 \end{pmatrix}}
\newcommand{\vektorz}[2]{\begin{pmatrix} #1 \\ #2 \end{pmatrix}}
\newcommand{\Mat}[9]{\begin{pmatrix}#1&#2&#3\\#4&#5&#6\\#7&#8&#9\end{pmatrix}}
\newcommand{\Matz}[4]{\begin{pmatrix}#1&#2\\#3&#4\end{pmatrix}}
\newcommand{\e}[1]{\,\si{#1}}
 


\begin{document}

	\FrontMatter
	% coordinates for background border
\newcommand{\diameter}{20}
\newcommand{\xone}{-15}
\newcommand{\xtwo}{160}
\newcommand{\yone}{15}
\newcommand{\ytwo}{-253}




\begin{titlepage}
    % background border
    \begin{tikzpicture}[overlay]
    \draw[color=gray]
            (\xone mm, \yone mm)
      -- (\xtwo mm, \yone mm)
    arc (90:0:\diameter pt)
      -- (\xtwo mm + \diameter pt , \ytwo mm)
        -- (\xone mm + \diameter pt , \ytwo mm)
    arc (270:180:\diameter pt)
        -- (\xone mm, \yone mm);
    \end{tikzpicture}



    % KIT image and sign for faculty of physics
    \begin{textblock}{10}[0,0](4.5,2.5)
        \includegraphics[width=.25\textwidth]{include/kitlogo.pdf}
    \end{textblock}
    

    % horizontal line
    \begin{textblock}{10}[0,0](4.2,3.1)
        \begin{tikzpicture}[overlay]
        \draw[color=gray]
                (\xone mm + 5 mm, -12 mm)
          -- (\xtwo mm + \diameter pt - 5 mm, -12 mm);
        \end{tikzpicture}
    \end{textblock}



    % begin of text part
    \changefont{phv}{m}{n}    % helvetica
    \centering



    % thesis topic (en and ge)
    \vspace*{3cm}
    \Huge\praktikum\\



    % author name and institute
    \vspace*{5cm}
    
    \huge\topic\\






    % examiners (Referenten)
    \vspace*{3cm}
    \Large
    \begin{center}
        \begin{tabular}[ht]{l c l } 
  \autora & \hfill & \textit{\maila} \\
\autorb & \hfill & \textit{\mailb} \\
        
        \end{tabular}
    \end{center}



    % working time
    \vspace{2cm}
    \begin{center}
        \large{Durchgeführt am}: \ptime
    \end{center}



    % lowest text blocks concerning the KIT
    \begin{textblock}{10}[0,0](4,16.8)
        \tiny{KIT -- Universität des Landes Baden-Württemberg und nationales %
              Forschungszentrum in der Helmholtz-Gemeinschaft}
    \end{textblock}
    \begin{textblock}{10}[0,0](14,16.75)
        \large{\textbf{www.kit.edu}}
    \end{textblock}
\end{titlepage}

	\tableofcontents                  
	\newpage
	\MainMatter

%Protokollstart

\chapter{Theoretische Grundlagen}

\section{Ziel des Experiments}

In dem Experiment des Quantenradierers werden quantenmechanische Eigenschaften wie die Aufenthaltswarscheinlichkeit anschaulich dargestellt. Hierzu werden kohärente Photonen zur Interferenz im Mach-Zehnder-Interferometer gebracht, wobei die herkömmliche Wellenmechanik an ihre Grenzen stößt. 

\section{Durchführung}
\begin{figure}[H]
	\begin{center}
		\includegraphics[width=0.8\textwidth]{images/Beamsplit.png}
		\caption{Mach-Zender-Interferometer, Quelle: Quantum Eraser, KSOP Optics \& Photonics Lab}
		\label{Mach-Zehnder}
	\end{center}
\end{figure}

In Abbildung \ref{Mach-Zehnder} ist ein Mach-Zehnder-Interferometer dargestellt. Ein Lichtstrahl wird dabei durch einen ersten Beamsplitter aufgeteilt. Der erste Strahl verläuft geradlinig weiter, der zweite wird durch zwei Spiegeln auf einen um $\Delta z/2$ längeren Umweg geführt. Nach dem Umweg werden die Teilstrahlen durch einen zweiten Beamsplitter wieder zusammengeführt und verlaufen weiter in gemeinsamer Bahn. Als Lichtquelle wird ein Laser verwendet, somit sind die Teilstrahlen kohärent und gut fokusiert. Nach dem zweiten Beamsplitter bilden sich zwei Richtungen für die rekombinierten Strahlen aus, beide werden auf seperaten Schirmen abgebildet, wie in Abb. \ref{Aufbau} ersichtlich. 
\begin{figure}[H]
	\begin{center}
		\includegraphics[width=0.8\textwidth]{images/Aufbau.png}
		\caption{Experimenteller Aufbau, Quelle: Quantum Eraser, KSOP Optics \& Photonics Lab}
		\label{Aufbau}
	\end{center}
\end{figure}
\subsection{Wellendynamik}
Die Photonendichte ist proportional zur Lichtintensität und gegeben mit:
\begin{equation}
n_{ph}(z,t)=\frac{I(\vec{x},t)}{\hbar \omega}=\frac{\vert \vec{E}\vert ^2}{\hbar \omega}
\end{equation}
Geht man von idealen Bedingungen aus, sollte klassisch gesehen die Teilchendichte sich nach aufteilen und wieder zusammenführen eines Strahls nicht verändern. Bei dem Interferometer haben wir aber einen Gangunterschied in den Teilstrahlen, welcher wie folgt erfasst werden kann:
\begin{align}
\si{Strahl\, A}\quad \vec{E}_A(z,t)&=\frac{1}{4}\vec{E_0}e^{i(k_z z-\omega t)}\\
\si{Strahl\, B}\quad \vec{E}_B(z,t)&=\frac{1}{4}\vec{E_0}e^{i(k_z (z+\Delta z)-\omega t)}
\end{align}
Hierbei ist die Feldstärke auf ein Viertel abgesunken, da sie bei zwei Beamsplittern jeweils halbiert wurde. Nach zusammenführen der Teilstrahlen bekommt man die überlagerte Welle als Superpossition:
\begin{equation}
\vec{E}_{ges}(z,t)= \vec{E}_1+\vec{E}_2=\frac{1}{4}\vec{E_0}(e^{ik_z z}+e^{ik_z (z+\Delta z)})e^{i\omega t}
\end{equation}
Daraus ergibt sich die Photonendichte für einen der beiden resultierenden Teilstrahlen hinter dem zweiten Beamsplitter zu:
\begin{equation}
n_{ph}(z,t)=\frac{\vert \vec{E}_{ges} \vert ^2}{\hbar \omega}=\frac{\vert \vec{E}_0\vert ^2}{4\hbar\omega}(1+\cos(k_z\Delta z))\label{n}
\end{equation}
Klassisch ist der $(1+\cos(k_z z))$-Term nicht erklärbar, das Ergebnis der Photonendichte wäre $\frac{\vert \vec{E}_0\vert}{2\hbar \omega}$ also gerade die Hälfte des Ausgangswertes, da nach dem zweiten Beamsplit der Grundstrahl effektiv in Zwei geteilt wurde. 
\section{Quantenradierer}
Im Experiment wird nun das Mach-Zehnder-Interferomenter wie in Abb. \ref{Mach-Zehnder} dargestellt, mit einen Polarisationsfilter pro Strahlgang ausgebaut. Geht man von einer in X-Richtung polarisierten Quelle aus und polarisiert die Strahlen in $\pm 45^\circ$ bekommt man für die Teilstrahlen hinter dem zweiten Beamsplitter:
\begin{align}
\vec{E}_A&=\frac{E_0}{2}e^{i(k_z z - \omega t)} \cdot\frac{1}{\sqrt{2}}\left(\begin{array}{c} 1 \\ 1 \end{array}\right)\\
\vec{E}_B&=\frac{E_0}{2}e^{i(k_z (z+\Delta z) - \omega t)}\cdot \frac{1}{\sqrt{2}}\left(\begin{array}{c} 1 \\ -1 \end{array}\right)
\end{align}
Berechnet man nun für den rekombinierten Strahl die Photonendichte nach \ref{n}, so verschwindet durch die Orthogonalität der Teilstrahlen das Betragsquadrat. Damit fällt Teilchendichte auf Null und auf beiden Schirmen kann keine Intensität festgestellt werden.\\
Nun wird zwischen den zweiten Splitter und einem Schirmen S1 ein zusätzlicher Polarisator, der Quantenradierer positioniert. Dieser Polarisator wird mit $0^\circ$ respektive zur Polarisation der Quelle eingestellt. Hierdurch werden die Teilstrahle erneut rotiert und mit einem Faktor $\frac{1}{\sqrt{2}}\left(\begin{array}{c} 1 \\ 0 \end{array}\right)$ multipliziert. Das Resultat lautet dann:
\begin{align}
\vec{E}_A&=\frac{E_0}{4}e^{i(k_z z - \omega t)} \cdot\left(\begin{array}{c} 1 \\ 0 \end{array}\right)\\
\vec{E}_B&=\frac{E_0}{4}e^{i(k_z (z+\Delta z) - \omega t)}\cdot \left(\begin{array}{c} 1 \\ 0 \end{array}\right)
\end{align}
Diese lineare Ausrichtung kann nun miteinander interferieren und für die Photonendichte ergibt sich:
\begin{equation}
n_{ph}(z,t)=\frac{\vert \vec{E}_0\vert ^2}{8\hbar\omega}(1+\cos(k_z\Delta z))
\end{equation}

\section{Gausssche Fehlerfortpflanzung}
Berechnet sich eine Größe $A$ aus mehreren Größen $x_i$ welche eine statistische Unsicherheit $\sigma_i$ besitzen, so propagiert die Unsicherheit der $x_i$ weiter zu einer Unsicherheit $\sigma_A$ in $A$. Diese Fehlerpropagation wird durch die Gaussche Fehlerfortpflanzug ausgedrückt:
\begin{equation}
\sigma_A = \sqrt{\sum_i \left( \frac{\p A}{\p x_i} \cc \sigma_i \right)^2 }
\label{gauss}
\end{equation}
\end{document}
