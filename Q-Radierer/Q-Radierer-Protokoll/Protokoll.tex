
\documentclass{include/thesisclass3}

\SelectLanguage{ngerman}
\usepackage{float}


% Titlepage settings
\newcommand{\praktikum}{Praktikum moderne Physik}
\newcommand{\autora}{Jens Schäfer}
\newcommand{\autorb}{Jan van der Linden}
\newcommand{\maila}{ugecd@student.kit.edu}
\newcommand{\mailb}{jan.vdlinden95@gmail.com}
\newcommand{\topic}{Quantenradierer}
\newcommand{\ptime}{22. Mai 2017}


% Shortcuts
\newcommand{\cc}{\cdot}
\newcommand{\rk}{\rangle}
\newcommand{\lk}{\langle}
\newcommand{\df}{\rightarrow}
\newcommand{\la}{\lambda}
\newcommand{\dd}{{\rm d}}
\newcommand{\ehm}{\mathbbm{1}}
\newcommand{\p}{\partial}
\newcommand{\soll}{\overset{!}{=}}
\newcommand{\D}{\Delta}
\newcommand{\eps}{\epsilon}
\newcommand{\vektor}[3]{\begin{pmatrix} #1 \\ #2 \\ #3 \end{pmatrix}}
\newcommand{\vektorz}[2]{\begin{pmatrix} #1 \\ #2 \end{pmatrix}}
\newcommand{\Mat}[9]{\begin{pmatrix}#1&#2&#3\\#4&#5&#6\\#7&#8&#9\end{pmatrix}}
\newcommand{\Matz}[4]{\begin{pmatrix}#1&#2\\#3&#4\end{pmatrix}}
\newcommand{\e}[1]{\,\si{#1}}
 


\begin{document}

	\FrontMatter
	% coordinates for background border
\newcommand{\diameter}{20}
\newcommand{\xone}{-15}
\newcommand{\xtwo}{160}
\newcommand{\yone}{15}
\newcommand{\ytwo}{-253}




\begin{titlepage}
    % background border
    \begin{tikzpicture}[overlay]
    \draw[color=gray]
            (\xone mm, \yone mm)
      -- (\xtwo mm, \yone mm)
    arc (90:0:\diameter pt)
      -- (\xtwo mm + \diameter pt , \ytwo mm)
        -- (\xone mm + \diameter pt , \ytwo mm)
    arc (270:180:\diameter pt)
        -- (\xone mm, \yone mm);
    \end{tikzpicture}



    % KIT image and sign for faculty of physics
    \begin{textblock}{10}[0,0](4.5,2.5)
        \includegraphics[width=.25\textwidth]{include/kitlogo.pdf}
    \end{textblock}
    

    % horizontal line
    \begin{textblock}{10}[0,0](4.2,3.1)
        \begin{tikzpicture}[overlay]
        \draw[color=gray]
                (\xone mm + 5 mm, -12 mm)
          -- (\xtwo mm + \diameter pt - 5 mm, -12 mm);
        \end{tikzpicture}
    \end{textblock}



    % begin of text part
    \changefont{phv}{m}{n}    % helvetica
    \centering



    % thesis topic (en and ge)
    \vspace*{3cm}
    \Huge\praktikum\\



    % author name and institute
    \vspace*{5cm}
    
    \huge\topic\\






    % examiners (Referenten)
    \vspace*{3cm}
    \Large
    \begin{center}
        \begin{tabular}[ht]{l c l } 
  \autora & \hfill & \textit{\maila} \\
\autorb & \hfill & \textit{\mailb} \\
        
        \end{tabular}
    \end{center}



    % working time
    \vspace{2cm}
    \begin{center}
        \large{Durchgeführt am}: \ptime
    \end{center}



    % lowest text blocks concerning the KIT
    \begin{textblock}{10}[0,0](4,16.8)
        \tiny{KIT -- Universität des Landes Baden-Württemberg und nationales %
              Forschungszentrum in der Helmholtz-Gemeinschaft}
    \end{textblock}
    \begin{textblock}{10}[0,0](14,16.75)
        \large{\textbf{www.kit.edu}}
    \end{textblock}
\end{titlepage}

	\tableofcontents                  
	\newpage
	\MainMatter

%Protokollstart

\chapter{Theoretische Grundlagen}
\section{Ziel des Experiments}

\begin{figure}[ht]
	\begin{center}
		%\includegraphics{images/experiment.png}
		\caption{Experimenteller Aufbau}
		\label{aufbau}
	\end{center}
\end{figure}

\end{document}
