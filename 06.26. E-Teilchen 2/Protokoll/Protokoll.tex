
\documentclass{include/thesisclass3}

\SelectLanguage{ngerman}
\usepackage{float}      

% Titlepage settings
\newcommand{\praktikum}{Praktikum moderne Physik}
\newcommand{\autora}{Jens Schäfer}
\newcommand{\autorb}{Jan van der Linden}
\newcommand{\maila}{ugecd@student.kit.edu}
\newcommand{\mailb}{jan.vdlinden95@gmail.com}
\newcommand{\topic}{Elementarteilchen}
\newcommand{\ptime}{26. Juni 2017}


% Shortcuts
\newcommand{\cc}{\cdot}
\newcommand{\rk}{\rangle}
\newcommand{\lk}{\langle}
\newcommand{\df}{\rightarrow}
\newcommand{\la}{\lambda}
\newcommand{\dd}{{\rm d}}
\newcommand{\ehm}{\mathbbm{1}}
\newcommand{\p}{\partial}
\newcommand{\soll}{\overset{!}{=}}
\newcommand{\D}{\Delta}
\newcommand{\eps}{\epsilon}
\newcommand{\vektor}[3]{\begin{pmatrix} #1 \\ #2 \\ #3 \end{pmatrix}}
\newcommand{\vektorz}[2]{\begin{pmatrix} #1 \\ #2 \end{pmatrix}}
\newcommand{\Mat}[9]{\begin{pmatrix}#1&#2&#3\\#4&#5&#6\\#7&#8&#9\end{pmatrix}}
\newcommand{\Matz}[4]{\begin{pmatrix}#1&#2\\#3&#4\end{pmatrix}}
\newcommand{\e}[1]{\,\si{#1}}
\newcommand{\listb}[1]{[$#1^- $,\,$#1^+ $];\quad}
\newcommand{\listx}[1]{[$#1^- $,\,$#1^+ $]\quad}
\newcommand{\listc}[1]{[$\nu_{#1} $,\,$\bar{\nu_{#1}}];\quad $}
\newcommand{\listd}[1]{[$\nu_{#1} $,\,$\bar{\nu_{#1}}]$}
\newcommand{\lista}[3]{\listb{#1 }\listb{#2 }\listb{#3 }\listc{#1}\listc{#2}\listd{#3}}
\newcommand{\listf}[1]{[$#1$,\,$\bar{#1} $];\quad}
\newcommand{\listg}[1]{[$#1$,\,$\bar{#1} $]}
\newcommand{\listq}[5]{\listf{#1 }\listf{#2 }\listf{#3 }\listf{#4}\listg{#5}}


\begin{document}

	\FrontMatter
	% coordinates for background border
\newcommand{\diameter}{20}
\newcommand{\xone}{-15}
\newcommand{\xtwo}{160}
\newcommand{\yone}{15}
\newcommand{\ytwo}{-253}




\begin{titlepage}
    % background border
    \begin{tikzpicture}[overlay]
    \draw[color=gray]
            (\xone mm, \yone mm)
      -- (\xtwo mm, \yone mm)
    arc (90:0:\diameter pt)
      -- (\xtwo mm + \diameter pt , \ytwo mm)
        -- (\xone mm + \diameter pt , \ytwo mm)
    arc (270:180:\diameter pt)
        -- (\xone mm, \yone mm);
    \end{tikzpicture}



    % KIT image and sign for faculty of physics
    \begin{textblock}{10}[0,0](4.5,2.5)
        \includegraphics[width=.25\textwidth]{include/kitlogo.pdf}
    \end{textblock}
    

    % horizontal line
    \begin{textblock}{10}[0,0](4.2,3.1)
        \begin{tikzpicture}[overlay]
        \draw[color=gray]
                (\xone mm + 5 mm, -12 mm)
          -- (\xtwo mm + \diameter pt - 5 mm, -12 mm);
        \end{tikzpicture}
    \end{textblock}



    % begin of text part
    \changefont{phv}{m}{n}    % helvetica
    \centering



    % thesis topic (en and ge)
    \vspace*{3cm}
    \Huge\praktikum\\



    % author name and institute
    \vspace*{5cm}
    
    \huge\topic\\






    % examiners (Referenten)
    \vspace*{3cm}
    \Large
    \begin{center}
        \begin{tabular}[ht]{l c l } 
  \autora & \hfill & \textit{\maila} \\
\autorb & \hfill & \textit{\mailb} \\
        
        \end{tabular}
    \end{center}



    % working time
    \vspace{2cm}
    \begin{center}
        \large{Durchgeführt am}: \ptime
    \end{center}



    % lowest text blocks concerning the KIT
    \begin{textblock}{10}[0,0](4,16.8)
        \tiny{KIT -- Universität des Landes Baden-Württemberg und nationales %
              Forschungszentrum in der Helmholtz-Gemeinschaft}
    \end{textblock}
    \begin{textblock}{10}[0,0](14,16.75)
        \large{\textbf{www.kit.edu}}
    \end{textblock}
\end{titlepage}

	\tableofcontents                  
	\newpage
	\MainMatter

%Protokollstart

\chapter{Theoretische Grundlagen}

\section{Das Standardmodell}
Das Standardmodell der Teilchenphysik ist das bisher vollständigste Modell zur Beschreibung der Teilchen und Wechselwirkungen auf elementarer Ebene.
Im Standardmodell werden alle bekannten Wechselwirkungen bis auf die Gravitation sehr genau beschrieben.
Dies ermöglicht eine präzise theoretische Berechnung und Vorhersage teilchenphysikalischer Prozesse, welche sehr häufig durch verschiedenste experimentelle Messungen bestätigt werden.
\\
Materie besteht auf elementarer Ebene aus wenigen verschiedenen Gruppen an Teilchen.
Eine der Gruppen sind die Quarks, sie können in drei Generationen angeordnet werden, wobei jeder Generation zwei Quarks zugeordnet werden. 
Eine weitere Gruppe, die Leptonen, lassen sich ebenso in drei Generationen anordnen, wobei eine Generation jeweils aus einem geladenen Lepton und einem neutralen Neutrino besteht.
Quarks, sowie Leptonen sind Fermionen mit halbzahligem Spin-$\frac{1}{2}$. 
Zu jedem der Quarks und Leptonen gibt es ein Antiteilchen mit gleicher Masse und entgegengesetzter Ladung.\\
Alle Quarks und Leptonen unterliegen der schwachen Wechselwirkung, die geladenen Leptonen und alle Quarks unterliegen ebenso der elektromagnetischen Wechselwirkung.
Da Leptonen keine Farbladung tragen, interagieren sie nicht über die starke Wechselwirkung, dies wird allein bei Quarks beobachtet.\\
Die Austauschteilchen der Wechselwirkungen sind Eichbosonen mit Spin-$1$.
Das Photon~$\gamma$ ist das Austauschteilchen für die elektromagnetische Wechselwirkung, für die starke Wechselwirkung sind dies die acht Gluonen $g$ und für die geladenen Ströme der schwachen Wechselwirkung die $W^\pm$-Bosonen sowie für die neutralen Ströme das $Z$-Boson.
Zusätzlich existiert noch ein weiteres Boson, das Higgs-Boson mit Spin-$0$.


\section{LEP und DELPHI}
Der LEP (Large Elektron-Positron-Collider) war ein am CERN befindliches Synchrotron für Elektron-Positron Kollisionen mit einem Umfang von fast 28km. Nach einem Umbau ist die Anlage heute als LHC bekannt und ist auf Hadronen ausgelegt. Durch Vorbeschleuniger werden die Leptonen in das Synchrotron eingeführt, wo sie durch Hochfrequenzkavitäten auf Schwerpunktsenergien von bis zu $\sqrt{s}=209\e{GeV}$ weiter beschleunigt wurden. Dabei werden die gleichschweren Positronen und Elektronen in entgegengesetzter Richtung beschleunigt. Dadurch können die selben Magnete zur Spurhaltung eingesetzt werden. Durch Ablenkmagnete werden die Strahlen schließlich in Detektoren wie dem DELPHI (\textit{detector with lepton, photon and hadron identification}) zur Kollision gebracht. 
Der Kern des Detektors ist von einem supraleitenden Solenoid eingefasst, der das Detektorinnere in ein starkes, homogenes Magnetfeld einschließt. 
Geladene Teilchen werden darin durch die Lorentzkraft auf gekrümmte Bahnen gelenkt. 
Über die Vermessung der gekrümmten Trajektorie erhält man Informationen über Ladung und Impuls der Teilchen. 
Dazu dient der im Inneren installierte Spurdetektor (\textit{tracker}). 
Zwischen Spurdetekor und Solenoid befinden sich das elektromagnetische und das hadronische Kaloriemeter in denen die meisten Teilchen destruktiv nach ihrer Energie analysiert werden. 
Außerhalb des Solenoids befinden sich Myonenkammern und ein Rückführjoch aus Eisen für das Magnetfeld. 

Die untersuchten Kollision zwischen Elektronen und ihren Antiteilchen resultiert in abzählbaren Kanälen, wobei folgende Bedingungen erfüllt sein müssen:
\begin{itemize}
\item Ladungszahlerhaltung\\
Da das Elektron und das Positron entgegengesetzt geladen sind, müssen die Produkte in der Summe eine verschwindende Ladung haben. Dadurch ist als virtuelles Teilchen das $W^+$- sowie das $W^-$-Boson nicht erlaubt. 

\item Leptonenfamilienzahlerhaltung\\
Die Leptonenzahl ist wie die Ladungszahl in Summe verschwindend, ebenso muss diese Zahl in den Produkten in der Summe Null sein. Damit und mit der Ladungserhaltungsbedingung reduzieren sich die leptonischen Ausgänge auf ein Lepton mit seinem zugehörigen Antilepton.

\item Energieerhaltung\\
Zunächst ist es erlaubt, dass der Zerfall des Z-Bosons auch in Kanälen endet, die aus mehreren Kanälen zusammengesetzt sind, solange keiner der Erhaltungssätze verletzt wird. Zum Beispiel kann ein vielfaches an \listb{e} entstehen. Dies ist allerdings durch die Energieerhaltung begrenzt. 

\item Impulserhaltung\\
Dies bedingt nicht direkt die Zerfallskanäle, jedoch die Richtungen, in denen die Produkte emittiert werden.

\item Erhaltung der Farbladung\\
Da das Edukt farbneutral ist, müssen die Produkte entweder im einzelnen farbneutral also Leptonen bzw. Photonen sein, alternativ können sie auch in ihrer Summe farbneutral sein, was die Kanäle über die Quarks eröffnet. 
\end{itemize}
Die erlaubten Kanäle lauten somit:\\
\lista{e}{\mu}{\tau};$ \quad [ \gamma ]; $\\
\listq{u}{d}{s}{c}{b}

Die Produkte können nach den selben Regeln selbst weiter zerfallen. So eröffnen sich für den Kanal \listx{ \tau } weiterführende geladene Kanäle die durch die hohe Energie des Lepton eine große Zerfallsbreite haben. Zerfälle in Top-Quarks sind durch die zu große Ruhemasse nicht erlaubt.



\chapter{Versuch}
In diesem Versuch sollen 1000 Detektorereignisse des DELPHI-Detektors analysiert und klassifiziert werden. 
Mit diesen klassifizierten Ereignissen sollen verschiedene Größen ermittelt werden.

\section{Verzweigungsverhältnis und Anzahl der Farbladungen}
Die Detektorereignisse sollen als $2$-, $3$- und Mehr-Jet-Ereignisse und leptonische Ereignisse klassifiziert werden. Bei den leptonischen Ereignissen soll zusätzlich die Leptonenfamilie bestimmt werden.\\
Aus dieser Klassifizierung kann das Verzweigungsverhältnis $R$ zwischen hadronischer und leptonischer Zerfallsbreite bestimmt werden
\[ R = \frac{ \Gamma_{had}}{\Gamma_{l\bar l}}\]
Da die Zerfallsbreiten direkt nicht bestimmbar sind aber allgemein proportional zur Ereignisanzahl sind, kann das Verhältnis der Zerfallsbreiten aus den Ereigniszahlen bestimmt werden.\\
Weiterhin soll die Leptonenuniversalität überprüft werden, wonach die Zerfallsbreiten der verschiedenen Leptonen gleich groß ist
\[ \Gamma_{l\bar l} = \Gamma_{ee} = \Gamma_{\mu\mu} = \Gamma_{\tau\tau}\]
Mit dem Verzweigungsverhältnis $R$ kann die Anzahl der verschiedenen Farbladungen $N_c$ über
\begin{equation}
 \Gamma_{had} = N_c \left( N_u \cc \Gamma_{u\bar u}^{SM} + N_d \cc \Gamma_{d \bar d}^{SM}\right)
\label{farbe}
\end{equation}
berechnet werden, wobei $N_d = 3$ und $N_u = 2$ gilt, da ein Zerfall nach $t \bar t$ nicht erlaubt ist und damit nur der Zerfall in zwei up-artige Quarks möglich ist. 

\section{Kopplungskonstante der starken Wechselwirkung}
Aus der Rate der $3$-Jet-Ereignisse im Verhältnis zu allen Jet-Ereignissen kann die Kopplungskonstante der starken Wechselwirkung approximiert werden. Als ein $3$-Jet-Ereignis wird hier ein Ereignis bezeichnet dessen invariante Masse der Parton-Paare eine Grenze $y_{cut}$ der Schwerpunktsenergie $\sqrt{s}$ übersteigt. Es ist $y_{cut}\cc \sqrt{s} = 0.02 \cc \sqrt{s}$ anzunehmen. Der Zusammenhang zwischen der Anzahl der $3$-Jet-Ereignisse und der Kopplungskonstante $\alpha_s$ ergibt sich dann aus
\[ \frac{N_3(y > y_{cut})}{N_{had}} \approx C(y > y_{cut}) \cc \alpha_s(m_z)\]
da für ein 3-Jet-Ereigniss eine zusätzliche Gluonenabstrahlung nötig ist und dieser Endzustand damit mit $\alpha_s$ unterdrückt ist.
Die Poportionalitätskonstante für $y_{cut} = 0.02$ ergibt sich zu $C = 2.72$. Da die Kopplungskonstante der starken Wechselwirkung laufend ist, das heißt mit der Masse skaliert, kann diese hier nur für die invariante Masse des $Z$-Bosons $m_Z$ untersucht werden.

\section{Anzahl der Neutrinofamilien}
Durch die Breite der $Z$-Resonanz $\Gamma_{tot}$ kann die Anzahl der Neutrinogenerationen $N_\nu$ bestimmt werden, da sich die gesamte Breite aus $\Gamma_{tot} = \Gamma_{had} + 3 \cc \Gamma_{l \bar l} + N_\nu \cc \Gamma_{\nu \bar \nu}$ zusammensetzt. Hierzu geht man vor wie folgt:\\
\begin{itemize}
\item Berechnen des hadronischen Wikrungsquerschnitts $\sigma_{had}$ aus der Luminosität $L$\\
$N_{had}=L\cdot \sigma_{had}$

\item Korrektur des Wirkungsquerschnitts durch Photonenemission\\
$\sigma_{had}'=\dfrac{\sigma_{had}}{1-0.263} $

\item Berechnen der totalen Zerfallsbreite $\Gamma_{tot}$ aus\\
$\sigma_{had}'=\dfrac{12 \pi \cdot \Gamma_{l\bar{l}}\cdot \Gamma_{had}}{m_Z^2\cdot \Gamma_{tot}^2}$
\item Berechnen der Zerfallsbreite der unsichtbaren Neutrino-Zerfälle $\Gamma_{inv}$\\
$\Gamma_{tot} = \Gamma_{had} + 3 \cc \Gamma_{l \bar l} + \Gamma_{inv}$

\item  Berechnen der Leptonenfamilienzahl\\
$N_\nu=\dfrac{\Gamma_{inv}}{\Gamma_{\nu_e\bar{\nu_e}}^{SM}}$
\end{itemize}
Im letzen Schritt wurde Leptonenuniversalität angenommen, daduch besteht die unsichtbare Zerfallsbreite zu gleichen Teilen aus den Zerfallsbreiten der einzelnen Neutrino-Paare. 


\chapter{Auswertung}
Die im vorigen Kapitel beschriebenen Verhältnisse werden nun mit den Werten aus den klassifizierten Ereignissen berechnet. Die Klassifikation der Zerfallskanäle der DELPHI-Ereignisse sind in Tabelle \ref{rohdaten} dargestellt.
\begin{table}[H]
\begin{center}
\begin{tabular}{   r | r| r | r | r | r | l   }
	Sheet & $e^-e^+$ & $\mu\bar{\mu}$ & $\tau\bar{\tau}$ & 2-Jet & 3-Jet & Total \\\hline
	1 & 10 & 4 & 11 & 47 & 28 & 100 \\ 
	2 & 5 & 4 & 7 & 58 & 26 & 100 \\ 
	3 & 7 & 2 & 2 & 55 & 34 & 100 \\ 
	4 & 6 & 8 & 6 & 50 & 30 & 100 \\ 
	5 & 5 & 3 & 6 & 52 & 34 & 100 \\ 
	6 & 7 & 7 & 5 & 53 & 28 & 100 \\ 
	7 & 8 & 4 & 4 & 46 & 38 & 100 \\ 
	8 & 7 & 2 & 9 & 52 & 30 & 100 \\ 
	9 & 5 & 3 & 6 & 51 & 35 & 100 \\ 
	10 & 10 & 5 & 5 & 50 & 30 & 100 \\ \hline
	Total &$N_{e \bar e} = 70$ & $N_{\mu \bar \mu} = 42$ &$N_{\tau \bar \tau} = 61$ & $N_2 = 514$ & $N_ 3 = 313$ & 1000
\end{tabular}
\end{center}
\caption{\label{rohdaten}Tabellarische Darstellung der klassifizierten Detektor-Ereignisse des DELPHI-Detektors. Die Ereignisse sind in Gruppen zu je 100 Ereignissen aufgeteilt.}
\end{table}
Da es sich bei den Werten um Zählvariablen handelt besitzen sie poissonverteilte Unsicherheiten $\sigma_{x_i}=\sqrt{x_i}$. Die auftretenden Abweichungen der berechneten Werte zu den Literaturwerten sind bedingt durch etwaige Interpretationsfehler einzelner Events sowie der Tatsache, dass bei nur 1000 Ereignissen durchaus hohe Fluktuationen möglich sind.
\section{Verzweigungsverhältnis}
Zur Bestimmung des Verzweigungsverhältnisses zwischen hadronischem und leptonischem Zerfall wird die durchschnittliche Anzahl der leptonischen Zerfälle pro Familie benötigt. Mit den klassifizierten Ereignissen ergibt sich dies zu
\[ N_{l \bar l} = \frac{N_{e\bar e} + N_{\mu \bar\mu} + N_{\tau \bar \tau}}{3} = 57.66 \pm 7.59 \pm 14.30\]
Die beiden angegebenen Unsicherheiten ergeben sich jeweils aus der Fortpflanzung der poissonischen Fehler der einzelnen Zählraten, sowie der Standardabweichung der Zählraten. 
Dieser zweite Fehler wurde explizit wegen der geforderten Leptonenuniversaltiät hinzugezogen, da in der Theorie gleiche Zählraten für die drei Leptonenfamilien zu erwarten wären.
Da jedoch ein signifikantes Defizit an $\mu \bar \mu$-Ereignissen im Vergleich zu den beiden anderen Ereignistypen beobachtet wurde, konnte die Leptonenuniversalität mit dieser Klassifikation nicht direkt bestätigt werden.
Für eine Sinnvolle Fehlerbetrachtung und Fehlerfortpflanzung wurde daher der zweite Fehler eingeführt, um den Zählratenunterschied auszugleichen.\\
Aus den Zählraten des hadronischen Jet-Ereignisse $N_{had} = 827\pm 29$ und der durchschnittlichen Zählrate der drei Leptonenfamilien $N_{l \bar l}$ ergibt sich nun das Verzweigungsverhältnis
\[ 
R = \frac{ N_{had}}{N_{l \bar l}} = 14.4 \pm 4.1\]
Der angegebene Fehler setzt sich aus der Gaussschen Fehlerfortpflanzung der Poissonfehler der Zählraten und der zusätzlichen Unsicherheit der Leptonenzählrate zusammen.

\section{Anzahl der Farbladungen}
Mit dem gegebenen Theoriewert $\Gamma^{SM}_{l \bar l} = 83.83\e{MeV}$ des Standardmodells kann aus dem Verzweigungsverhältnis die hadronische Zerfallsbreite bestimmt werden
\[\Gamma_{had} = R \cc \Gamma^{SM}_{l \bar l} =  (1202 \pm 340)\e{MeV}\]
Dieser Wert wird benötigt um die Anzahl der Farbladung mit Gleichung \ref{farbe} zu bestimmen:
\[ N_c = \frac{\Gamma_{had}}{2\cc \Gamma^{SM}_{u\bar u} + 3 \cc \Gamma^{SM}_{d \bar d}} = 2.07 \pm 0.59\]
Hier wurden die gegebenen Theoriewerte $\Gamma_{u\bar u}^{SM}=98.88\e{MeV}$ und $\Gamma_{d \bar d}^{SM}=127.48\e{MeV}$ des Standardmodells verwendet. Der wahre Wert $N_c^{SM} = 3$ liegt im $2\sigma$-Intervall des hier bestimmten Wertes. Grund für die hohe Abweichung könnte die Fehlklassifizierung einiger Ereignisse sein, als auch statistische Anomalien im betrachteten Datensatz von 1000 Ereignissen.

\section{Kopplungskonstante der starken Wechselwirkung}
Aus dem Verhältnis der klassifizierten $3$- und Mehr-Jet-Ereignissen $N_3 = 313 \pm 18$ zu allen hadronischen Ereignissen $N_{had} = 827 \pm 29$ kann mit der gegebenen Konstante $C = 2.72$ die Kopplungskonstante der starken Wechselwirkung $\alpha_s$ an der $Z$-Resonanz von $m_Z = 91.19\e{GeV}$ approximiert werden:
\[ \alpha_s(m_Z) = \frac{ N_3}{N_{had} \cc C} = 0.139\pm 0.009 \]
Der eigentliche Wert des Standardmodells\footnote{Quelle: Particle Data Group, \url{pdg.lpl.gov}} liegt bei $\alpha_s^{SM}(m_Z) = 0.118$ und ist damit über $2\sigma$ von dem hier bestimmten Wert entfernt.

\section{Anzahl der Neutrinofamilie}
Der betrachtete Datensatz entpsicht einer Luminosität von $L = 28.48\e{nb^{-1}}$, damit beträgt der hadronische Wirkungsquerschnitt
\[\sigma_{had} = \frac{ N_{had}}{L} = (29.04 \pm 5.39)\e{nb}\]
Mit dem angegebenen Faktor von $26.3\e{\%}$ für die Korrektur des Wirkungsquerschnittes auf Grund der Photonenemission erhöht sich der tatsächliche Wirkungsquerschnitt zu
\[\sigma_{had} = (39.40 \pm 6.28)\e{nb}\]
Damit kann die totale Zerfallsbreite des $Z$-Bosons berechnet werden
\[\Gamma_{tot} = \sqrt{ \frac{ 12 \pi \cc \Gamma_{l \bar l} \Gamma_{had}}{\sigma_{had} m_Z^2}} = (2127 \pm 345)\e{MeV}\]
Da somit die hadronische, die leptonische als auch die totale Zerfallsbreite bekannt sind kann auf die unsichtbare Zerfallsbreite der Neutrinos geschlossen werden
\[\Gamma_{inv} = \Gamma_{tot} - \Gamma_{had} - 3 \Gamma_{l\bar l} = (673\pm 485)\e{MeV}\]
Unter der Annahme der Leptonenuniversalität und der gegebene Zerfallsbreite des Elektronenneutrino-Kanals $\Gamma^{SM}_{\nu_e \bar \nu_e}=166.1\e{MeV}$ des Standardmodells ist somit die Anzahl der Neutrinofamilien berechenbar
\[N_\nu = \frac{\Gamma_{inv}}{\Gamma_{\nu_e \bar \nu_e}^{SM}} = 4.05\pm 2.92\]
Die Unsicherheit auf diesen Wert liegt in der selben Größenordnung wie der Wert selbst.

\end{document}
